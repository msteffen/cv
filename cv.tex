% Template taken from:
% https://www.overleaf.com/latex/templates/rendercv-sb2nov-theme/gdspgtsnfncm

\documentclass[10pt, letterpaper]{article}

\input{setup.tex}

\begin{document}

\newcommand{\AND}{\unskip
    \cleaders\copy\ANDbox\hskip\wd\ANDbox
    \ignorespaces
}

\newsavebox\ANDbox
\sbox\ANDbox{}

\placelastupdatedtext

\begin{header}
    \textbf{\fontsize{24 pt}{24 pt}\selectfont Matthew Steffen}

    \normalsize
    \textit{Leader with talents for communication and consensus-building. Interested in both Manager and I.C. roles.}

    \mbox{{\color{black}\footnotesize\faMapMarker*}\hspace*{0.13cm}San Carlos, CA}
    \kern 0.25 cm
    \AND
    \kern 0.25 cm
    \mbox{\hrefWithoutArrow{mailto:m@tthew.io}{\color{black}{\footnotesize\faEnvelope[regular]}\hspace*{0.13cm}m@tthew.io}}
    \kern 0.25 cm
    \AND
    \kern 0.25 cm
    \mbox{\hrefWithoutArrow{tel:+1-713-208-6899}{\color{black}{\footnotesize\faPhone*}\hspace*{0.13cm}713-208-6899}}
    \kern 0.25 cm
    \AND
    \kern 0.25 cm
    \mbox{\hrefWithoutArrow{https://prog.blog/}{\color{black}{\footnotesize\faLink}\hspace*{0.13cm}https://prog.blog}}
    \kern 0.25 cm
    \AND
    \kern 0.25 cm
    \mbox{\hrefWithoutArrow{https://linkedin.com/in/m12n}{\color{black}{\footnotesize\faLinkedinIn}\hspace*{0.13cm}m12n}}
    \kern 0.25 cm
    \AND
    \kern 0.25 cm
    \mbox{\hrefWithoutArrow{https://github.com/msteffen}{\color{black}{\footnotesize\faGithub}\hspace*{0.13cm}msteffen}}
\end{header}

%%%%%%%%%%%%%%%%%%%%%%%%%%%%%%%%%%%%%%%%%%%%%%%%%%%%%%%%%%%%%%%%
%                      Skills                                  %
%%%%%%%%%%%%%%%%%%%%%%%%%%%%%%%%%%%%%%%%%%%%%%%%%%%%%%%%%%%%%%%%
\section{Skills}

\begin{onecolentry}
    \begin{highlights}
        \item\textbf{Engineering:} Extensive professional experience with C++, Java, and Go codebases, plus smaller projects in Typescript+React and Python. I am always excited to learn new languages and tools.
        \item\textbf{Management:} Managed up to five engineers and co-managed seven, and I have hired into backend and full-stack roles. I have strong written and verbal communication skills and wrote many of Pachyderm's internal templates, handbooks and guides.
    \end{highlights}
\end{onecolentry}

%%%%%%%%%%%%%%%%%%%%%%%%%%%%%%%%%%%%%%%%%%%%%%%%%%%%%%%%%%%%%%%%
%                     Experience                               %
%%%%%%%%%%%%%%%%%%%%%%%%%%%%%%%%%%%%%%%%%%%%%%%%%%%%%%%%%%%%%%%%
\section{Experience}

%%%%%%%%%%%%%% HPE %%%%%%%%%%%%%%%%%%
\begin{twocolentry}{
    \textit{San Jose \& Remote}

    \textit{March 2023 – Nov 2024}
}
    \textbf{Hewlett Packard Enterprise (via Pachyderm acquisition)}

    \textit{Software Engineering Manager}
\end{twocolentry}

\vspace{0.1 cm}

\begin{onecolentry}
    \begin{highlights}
        \item I was designated a Key Employee when HPE acquired Pachyderm in 2023.
        \item Continued managing the Pachyderm Integrations Team through May 2024, shipping a complete rewrite of our Jupyter extension to meet the security requirements of HPE's customers.
        \item Co-managed a team of seven engineers, after HPE re-organized Pachyderm into this single team in June 2024. I entirely overhauled the team's processes and worked with IT to implement a common development environment. These efficiency improvements allowed us to ship several long-requested features.
    \end{highlights}
\end{onecolentry}

\vspace{0.2 cm}

%%%%%%%%%%%%%% Pachyderm %%%%%%%%%%%%%%%%%%
\begin{twocolentry}{
    \textit{San Francisco \& Remote}

    \textit{Jan 2022 – Feb 2023}
}
    \textbf{Pachyderm}

    \textit{Software Engineering Manager}
\end{twocolentry}

\vspace{0.1 cm}

\begin{comment}
\begin{onecolentry}
  %\begin{itemize}[
  %      topsep=0.10 cm,
  %      parsep=0.10 cm,
  %      partopsep=0pt,
  %      itemsep=0pt,
  %      label={},
  %      leftmargin=0.4 cm + 10pt
  %]
  %\textit{Pachyderm was a distributed filesystem with versioning, engineered for incremental batch data processing (e.g. ETL and ML preprocessing), and, eventually, AI training. We billed ourselves as ``Git for data''.}
  \small
  \textbf{Pachyderm is a distributed filesystem with versioning, designed to eliminate duplicate work in e.g. ETL and ML preprocessing pipelines and, eventually, AI training. We were ``Git for data'' or ``Bazel for data''.}
  %\end{itemize}
\end{onecolentry}

\vspace{0.1 cm}

% NB. comment these out above if including this blurb
\begin{twocolentry}{
    \textit{Jan 2022 – Feb 2023}
}
    \textit{Software Engineering Manager}
\end{twocolentry}

\vspace{0.1 cm}

\end{comment}

\begin{onecolentry}
    \begin{highlights}
        \item Hired for and supported the team through the design and development of our JupyterLab extension.
        \item This extension, our biggest project, was a developer tool that allowed data scientists to explore and consume Pachyderm data directly in a Jupyter notebook (which required deep modifications to Jupyter's frontend) and run Jupyter notebooks as sharded, orchestrated Pachyderm pipelines without modification.
        \item Contributed to design reviews and stayed in our on-call rotation throughout my tenure as manager.
    \end{highlights}
\end{onecolentry}

\vspace{0.2 cm}

\begin{twocolentry}{
    \textit{Aug 2016 – Dec 2021}
}
    \textit{Software Engineer}
\end{twocolentry}

\vspace{0.1 cm}

\begin{onecolentry}
    \begin{highlights}
        \item Pachyderm is a distributed filesystem with versioning that eliminates duplicate work in data pipelines and AI training. When a Pachyderm pipeline runs on a new version of its data, it only processes new and updated shards and re-uses previous results otherwise. In some cases, we reduced the cost of our customers' pipelines by over 90\%. We described ourselves as ``Git for data'' or, sometimes, ``Bazel for data''.
        \item Implemented several major features, including an authorization system and a complete rewrite of our pipeline controller, among others (Go).
        \item Started new Integrations Team, to build integrations between Pachyderm and other AI/ML products (Kubeflow, Seldon, LabelStudio, AzureML, and others). I acted as tech lead through its growth to five engineers.
    \end{highlights}
  \end{onecolentry}

\vspace{0.2 cm}

%%%%%%%%%%%%%% Google %%%%%%%%%%%%%%%%%%
\begin{twocolentry}{
    \textit{Mountain View, CA}

    \textit{Aug 2011 – July 2016}
}
    \textbf{Google}

    \textit{Software Engineer}
\end{twocolentry}

\vspace{0.1 cm}
\begin{onecolentry}
    \begin{highlights}
        \item Independently maintained a legacy authorization service used by Google Docs, Photos, YouTube, Search, and other products through 2012 (Java).
        \item Joined the \href{https://research.google/pubs/zanzibar-googles-consistent-global-authorization-system/}{Zanzibar} team in 2013 as the fourth engineer. I migrated users from the legacy service to Zanzibar and implemented features and performance improvements in Zanzibar itself (C++).
        \item Contributed features and performance improvements to Cloud IAM in late 2015 and 2016.
    \end{highlights}
\end{onecolentry}

%%%%%%%%%%%%%%%%%%%%%%%%%%%%%%%%%%%%%%%%%%%%%%%%%%%%%%%%%%%%%%%%
%                     Education                                %
%%%%%%%%%%%%%%%%%%%%%%%%%%%%%%%%%%%%%%%%%%%%%%%%%%%%%%%%%%%%%%%%
\section{Education}

\begin{twocolentry}{
    \textit{Sept 2007 – May 2011}
}
    \textbf{University of Chicago}
\end{twocolentry}

% This should be in the twocolentry above, but it doesn't all fit in
% the left column. Using a onecolentry here makes the spacing correct
% but keeps everything on one line
\begin{onecolentry}
    \textit{BS in Computer Science (also met requirements for BS in Mathematics)}
\end{onecolentry}

\begin{comment}

\vspace{0.1 cm}

\begin{onecolentry}
    \begin{highlights}
        \item ACM International Collegiate Programming Contest (ICPC), World Finalist (2010, 2011)
        \item USA Powerlifting, Collegiate National Finalist (2011)
    \end{highlights}
\end{onecolentry}
\end{comment}

\end{document}
