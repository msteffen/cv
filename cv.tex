% Template taken from:
% https://www.overleaf.com/latex/templates/rendercv-sb2nov-theme/gdspgtsnfncm

\documentclass[10pt, letterpaper]{article}

% Packages:
\usepackage[
    ignoreheadfoot, % set margins without considering header and footer
    top=2 cm, % seperation between body and page edge from the top
    bottom=2 cm, % seperation between body and page edge from the bottom
    left=2 cm, % seperation between body and page edge from the left
    right=2 cm, % seperation between body and page edge from the right
    footskip=1.0 cm, % seperation between body and footer
    % showframe % for debugging 
]{geometry} % for adjusting page geometry
\usepackage{titlesec} % for customizing section titles
\usepackage{tabularx} % for making tables with fixed width columns
\usepackage{array} % tabularx requires this
\usepackage[dvipsnames]{xcolor} % for coloring text
\definecolor{primaryColor}{RGB}{0, 79, 144} % define primary color
\usepackage{enumitem} % for customizing lists
\usepackage{fontawesome5} % for using icons
\usepackage{amsmath} % for math
\usepackage[
    pdftitle={John Doe's CV},
    pdfauthor={John Doe},
    pdfcreator={LaTeX with RenderCV},
    colorlinks=true,
    urlcolor=primaryColor
]{hyperref} % for links, metadata and bookmarks
\usepackage[pscoord]{eso-pic} % for floating text on the page
\usepackage{calc} % for calculating lengths
\usepackage{bookmark} % for bookmarks
\usepackage{lastpage} % for getting the total number of pages
\usepackage{changepage} % for one column entries (adjustwidth environment)
\usepackage{paracol} % for two and three column entries
\usepackage{ifthen} % for conditional statements
\usepackage{needspace} % for avoiding page brake right after the section title
\usepackage{iftex} % check if engine is pdflatex, xetex or luatex

% Ensure that generate pdf is machine readable/ATS parsable:
\ifPDFTeX
    \input{glyphtounicode}
    \pdfgentounicode=1
    % \usepackage[T1]{fontenc} % this breaks sb2nov
    \usepackage[utf8]{inputenc}
    \usepackage{lmodern}
\fi

% Some settings:
\AtBeginEnvironment{adjustwidth}{\partopsep0pt} % remove space before adjustwidth environment
\pagestyle{empty} % no header or footer
\setcounter{secnumdepth}{0} % no section numbering
\setlength{\parindent}{0pt} % no indentation
\setlength{\topskip}{0pt} % no top skip
\setlength{\columnsep}{0cm} % set column seperation
\makeatletter
\let\ps@customFooterStyle\ps@plain % Copy the plain style to customFooterStyle
\patchcmd{\ps@customFooterStyle}{\thepage}{
    \color{gray}\textit{\small John Doe - Page \thepage{} of \pageref*{LastPage}}
}{}{} % replace number by desired string
\makeatother
\pagestyle{customFooterStyle}

\titleformat{\section}{\needspace{4\baselineskip}\bfseries\large}{}{0pt}{}[\vspace{1pt}\titlerule]

\titlespacing{\section}{
    % left space:
    -1pt
}{
    % top space:
    0.3 cm
}{
    % bottom space:
    0.2 cm
} % section title spacing

\renewcommand\labelitemi{$\circ$} % custom bullet points
\newenvironment{highlights}{
    \begin{itemize}[
        topsep=0.10 cm,
        parsep=0.10 cm,
        partopsep=0pt,
        itemsep=0pt,
        leftmargin=0.4 cm + 10pt
    ]
}{
    \end{itemize}
} % new environment for highlights

\newenvironment{highlightsforbulletentries}{
    \begin{itemize}[
        topsep=0.10 cm,
        parsep=0.10 cm,
        partopsep=0pt,
        itemsep=0pt,
        leftmargin=10pt
    ]
}{
    \end{itemize}
} % new environment for highlights for bullet entries


\newenvironment{onecolentry}{
    \begin{adjustwidth}{
        0.2 cm + 0.00001 cm
    }{
        0.2 cm + 0.00001 cm
    }
}{
    \end{adjustwidth}
} % new environment for one column entries

\newenvironment{twocolentry}[2][]{
    \onecolentry
    \def\secondColumn{#2}
    \setcolumnwidth{\fill, 4.5 cm}
    \begin{paracol}{2}
}{
    \switchcolumn \raggedleft \secondColumn
    \end{paracol}
    \endonecolentry
} % new environment for two column entries

\newenvironment{header}{
    \setlength{\topsep}{0pt}\par\kern\topsep\centering\linespread{1.5}
}{
    \par\kern\topsep
} % new environment for the header

\newcommand{\placelastupdatedtext}{% \placetextbox{<horizontal pos>}{<vertical pos>}{<stuff>}
  \AddToShipoutPictureFG*{% Add <stuff> to current page foreground
    \put(
        \LenToUnit{\paperwidth-2 cm-0.2 cm+0.05cm},
        \LenToUnit{\paperheight-1.0 cm}
    ){\vtop{{\null}\makebox[0pt][c]{
        \small\color{gray}\textit{Last updated in September 2024}\hspace{\widthof{Last updated in September 2024}}
    }}}%
  }%
}%

% save the original href command in a new command:
\let\hrefWithoutArrow\href

% new command for external links:
\renewcommand{\href}[2]{\hrefWithoutArrow{#1}{\ifthenelse{\equal{#2}{}}{ }{#2 }\raisebox{.15ex}{\footnotesize \faExternalLink*}}}




\begin{document}

\newcommand{\AND}{\unskip
    \cleaders\copy\ANDbox\hskip\wd\ANDbox
    \ignorespaces
}

\newsavebox\ANDbox
\sbox\ANDbox{}

\placelastupdatedtext

\begin{header}
    \textbf{\fontsize{24 pt}{24 pt}\selectfont Matthew Steffen}

    \normalsize
    \textit{Leader with talents for communication and consensus-building. Open to both Manager and I.C. roles.}

    \mbox{{\color{black}\footnotesize\faMapMarker*}\hspace*{0.13cm}San Carlos, CA}
    \kern 0.25 cm
    \AND
    \kern 0.25 cm
    \mbox{\hrefWithoutArrow{mailto:m@tthew.io}{\color{black}{\footnotesize\faEnvelope[regular]}\hspace*{0.13cm}m@tthew.io}}
    \kern 0.25 cm
    \AND
    \kern 0.25 cm
    \mbox{\hrefWithoutArrow{tel:+1-713-208-6899}{\color{black}{\footnotesize\faPhone*}\hspace*{0.13cm}713-208-6899}}
    \kern 0.25 cm
    \AND
    \kern 0.25 cm
    \mbox{\hrefWithoutArrow{https://prog.blog/}{\color{black}{\footnotesize\faLink}\hspace*{0.13cm}https://prog.blog}}
    \kern 0.25 cm
    \AND
    \kern 0.25 cm
    \mbox{\hrefWithoutArrow{https://linkedin.com/in/m12n}{\color{black}{\footnotesize\faLinkedinIn}\hspace*{0.13cm}m12n}}
    \kern 0.25 cm
    \AND
    \kern 0.25 cm
    \mbox{\hrefWithoutArrow{https://github.com/msteffen}{\color{black}{\footnotesize\faGithub}\hspace*{0.13cm}msteffen}}
\end{header}

\begin{comment}
    \vspace{0.3 cm - 0.3 cm}
    \section{Welcome to RenderCV!}

    \begin{onecolentry}
        \href{https://rendercv.com}{RenderCV} is a LaTeX-based CV/resume version-control and maintenance app. It allows you to create a high-quality CV or resume as a PDF file from a YAML file, with \textbf{Markdown syntax support} and \textbf{complete control over the LaTeX code}.
    \end{onecolentry}

    \vspace{0.2 cm}
    \begin{onecolentry}
        The boilerplate content was inspired by \href{https://github.com/dnl-blkv/mcdowell-cv}{Gayle McDowell}.
    \end{onecolentry}

    \section{Quick Guide}
    
    \begin{onecolentry}
        \begin{highlightsforbulletentries}
        \item Each section title is arbitrary and each section contains a list of entries.
        \item There are 7 unique entry types: \textit{BulletEntry}, \textit{TextEntry}, \textit{EducationEntry}, \textit{ExperienceEntry}, \textit{NormalEntry}, \textit{PublicationEntry}, and \textit{OneLineEntry}.
        \item Select a section title, pick an entry type, and start writing your section!
        \item \href{https://docs.rendercv.com/user_guide/}{Here}, you can find a comprehensive user guide for RenderCV.
        \end{highlightsforbulletentries}
    \end{onecolentry}
\end{comment}

%%%%%%%%%%%%%%%%%%%%%%%%%%%%%%%%%%%%%%%%%%%%%%%%%%%%%%%%%%%%%%%%
%                      Skills                                  %
%%%%%%%%%%%%%%%%%%%%%%%%%%%%%%%%%%%%%%%%%%%%%%%%%%%%%%%%%%%%%%%%
\section{Skills}

\begin{onecolentry}
  \begin{itemize}[label={},itemindent=-0.4cm-10pt]
        \item\textbf{Engineering:} Written a significant amount of C++, Java, and Go professionally, and smaller projects in Typescript+React and Python. Excited to learn.

        \item\textbf{Management:} Managed up to five engineers and co-managed seven, and I have hired into backend and full-stack roles. I have strong written and verbal communication skills and wrote many of Pachyderm's internal templates, handbooks and guides.
    \end{itemize}
\end{onecolentry}

%%%%%%%%%%%%%%%%%%%%%%%%%%%%%%%%%%%%%%%%%%%%%%%%%%%%%%%%%%%%%%%%
%                     Experience                               %
%%%%%%%%%%%%%%%%%%%%%%%%%%%%%%%%%%%%%%%%%%%%%%%%%%%%%%%%%%%%%%%%
\section{Experience}

%%%%%%%%%%%%%% HPE %%%%%%%%%%%%%%%%%%
\begin{twocolentry}{
    \textit{San Jose, CA}

    \textit{March 2023 – Nov 2024}
}
    \textbf{Hewlett Packard Enterprise (via Pachyderm acquisition)}

    \textit{Software Engineering Manager}
\end{twocolentry}

\vspace{0.1 cm}

\begin{onecolentry}
    \begin{highlights}
        \item Designated a Key Employee when Pachyderm was acquired by HPE in 2023.
        \item Continued managing the Pachyderm Integrations Team through May 2024, hiring a full-stack engineer and shipping a complete rewrite of our Jupyter extension to meet the security requirements of HPE's customers.
        \item Co-managed a team of seven engineers, after Pachyderm was re-oranized into this single team in June 2024.
        \item Led several internal efficiency improvements, such as codifying our engineering standards and expectations in an engineering handbook and standardizing on a single common development environment, which I worked with IT to implement. This streamlined process helped the team ship several long-requested features.
    \end{highlights}
\end{onecolentry}

\vspace{0.2 cm}

%%%%%%%%%%%%%% Pachyderm %%%%%%%%%%%%%%%%%%
\begin{twocolentry}{
    \textit{San Francisco, CA}

    \textit{Jan 2022 – Feb 2023}
}
    \textbf{Pachyderm}

    \textit{Software Engineering Manager}
\end{twocolentry}

\vspace{0.1 cm}

\begin{onecolentry}
    \begin{highlights}
        \item Managed our Integrations Team starting in 2022, hiring a backend engineer onto the team (a backfill) and supported the team through the design, development, and initial release of our JupyterLab extension. I continued contributing to design reviews and going on call through my tenure as manager.
        \item This extension allowed data scientists to explore and consume Pachyderm data in a Jupyter notebook, and convert Jupyter notebooks into Pachyderm pipelines. Using it, they could develop Pachyderm pipelines interactively, which made the product much more accessible to new users, and made Pachyderm pipelines much easier to debug.
    \end{highlights}
\end{onecolentry}

\vspace{0.2 cm}

\begin{twocolentry}{
    \textit{Aug 2016 – Dec 2021}
}
    \textit{Software Engineer}
\end{twocolentry}

\vspace{0.1 cm}

\begin{onecolentry}
    \begin{highlights}
        \item Implemented several major features, including an authorization system and a complete rewrite of our pipeline controller, among others (Go).
        \item Started new Integrations Team, to build integrations between Pachyderm and other AI/ML products. I acted as tech lead through its growth to five engineers.
        \item Wrote our first integration, with Kubeflow, and supported an overhaul of that initial Kubeflow integration as well as new integrations with AzureML, Sagemaker, Seldon, Spark, and MLFlow.
    \end{highlights}
  \end{onecolentry}

\vspace{0.2 cm}

%%%%%%%%%%%%%% Google %%%%%%%%%%%%%%%%%%
\begin{twocolentry}{
    \textit{Mountain View, CA}

    \textit{Aug 2011 – July 2016}
}
    \textbf{Google}

    \textit{Software Engineer}
\end{twocolentry}

\vspace{0.1 cm}
\begin{onecolentry}
    \begin{highlights}
        \item Independently maintained a legacy authorization service used by Google Docs, Photos, YouTube, Search, and other products through 2012 (Java).
        \item Joined the \href{https://research.google/pubs/zanzibar-googles-consistent-global-authorization-system/}{Zanzibar} team in 2013 as the fourth engineer. I migrated users from the legacy service to Zanzibar and implemented features and performance improvements in Zanzibar itself (C++).
    \end{highlights}
\end{onecolentry}

%%%%%%%%%%%%%%%%%%%%%%%%%%%%%%%%%%%%%%%%%%%%%%%%%%%%%%%%%%%%%%%%
%                     Education                                %
%%%%%%%%%%%%%%%%%%%%%%%%%%%%%%%%%%%%%%%%%%%%%%%%%%%%%%%%%%%%%%%%
\section{Education}

\begin{twocolentry}{
    \textit{Sept 2007 – May 2011}
}
    \textbf{University of Chicago}
\end{twocolentry}

% This should be in the twocolentry above, but it doesn't all fit in
% the left column. Using a onecolentry here makes the spacing correct
% but keeps everything on one line
\begin{onecolentry}
    \textit{BS in Computer Science (also met requirements for BS in Mathematics)}
\end{onecolentry}

\vspace{0.1 cm}
\begin{onecolentry}
    \begin{highlights}
        \item ACM International Collegiate Programming Contest (ICPC), World Finalist (2010, 2011)
        \item USA Powerlifting, Collegiate National Finalist (2011)
    \end{highlights}
\end{onecolentry}

\end{document}
